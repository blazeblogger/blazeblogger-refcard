% BlazeBlogger Quick Reference Card
% Copyright (C) 2009 Jaromir Hradilek

% Permission is granted to copy,  distribute  and/or  modify  this document
% under the terms of the GNU Free Documentation License, Version 1.3 or any
% later  version  published  by  the  Free  Software  Foundation;  with  no
% Invariant Sections, no Front-Cover Texts, and no Back-Cover Texts.
%
% A copy of the license  is included  as a file called COPYING  in the main
% directory of this document's source package.

\documentclass[a4paper,10pt,landscape,twoside]{article}

\usepackage[landscape,margin=1cm]{geometry}
\usepackage{multicol}

\newenvironment{examples}%
{\vspace{.5em}
 \noindent\begin{tabular}{p{.49\columnwidth}p{.40\columnwidth}}}
{\end{tabular}
 \vspace{.5em}}

\newcommand{\command}[2]{%
 \scriptsize\sffamily\bfseries #1 & \scriptsize\sffamily #2 \\}
\newcommand{\longcommand}[2]{%
 \multicolumn{2}{l}{\scriptsize\sffamily\bfseries #1} \\
 & \scriptsize\sffamily #2 \\}
\newcommand{\plaincommand}[1]{%
 \multicolumn{2}{l}{\scriptsize\sffamily\bfseries #1} \\}

\makeatletter
\renewcommand\section{\@startsection {section}{1}{\z@}%
                                     {-3.5ex \@plus -1ex \@minus -.2ex}%
                                     {2.3ex \@plus.2ex}%
                                     {\normalsize\bfseries}}
\makeatother

\pagestyle{empty}

\begin{document}

\footnotesize

\begin{multicols}{3}

\section*{Creating a New Blog}

\noindent Create a new blog repository:

\begin{examples}
  \command{blaze-init}{create it in the current directory}
  \command{blaze-init -{}-blogdir \~{}/public\_html}{create it in the selected directory}
  \command{blaze-init -{}-verbose}{list files as they are created}
\end{examples}

\noindent Recover corrupted repository:

\begin{examples}
  \command{blaze-init}{keep already existing files intact}
  \command{blaze-init -{}-force}{revert files to their initial state}
\end{examples}

\section*{Configuring the Blog}

\noindent Configure the blog:

\begin{examples}
  \command{blaze-config blog.title Faraway}{choose the blog title}
  \command{blaze-config blog.subtitle i had a dream}{choose the blog subtitle}
  \command{blaze-config blog.theme default.html}{choose the blog theme}
  \command{blaze-config blog.style default.css}{choose the blog style}
  \command{blaze-config blog.lang en\_GB}{choose the blog language}
  \command{blaze-config blog.posts 10}{list 10 posts per page}
  \longcommand{blaze-config blog.url http://blog.example.com/}{set the base URL}
\end{examples}

\noindent Configure the colours:

\begin{examples}
  \command{blaze-config color.list true}{use coloured repository listing}
  \command{blaze-config color.log true}{use coloured log listing}
\end{examples}

\noindent Configure the BlazeBlogger:

\begin{examples}
  \command{blaze-config core.editor nano}{set the external text editor}
  \command{blaze-config core.encoding ISO-8859-1}{set the correct code page}
  \command{blaze-config core.extension php}{set the file extension}
  \longcommand{blaze-config core.processor `markdown -{}-html4tags \%in\% $>$ \%out\%'}{choose other markup language}
\end{examples}

\noindent Configure the post information:

\begin{examples}
  \command{blaze-config post.author none}{location of the post author}
  \command{blaze-config post.date top}{location of the date of publishing}
  \command{blaze-config post.tags bottom}{location of the post tags}
\end{examples}

\noindent Configure the blog author:

\begin{examples}
  \command{blaze-config user.name David}{set author's name}
  \longcommand{blaze-config user.email david@example.com}{set author's e-mail}
\end{examples}

\section*{Adding New Content}

\noindent Add new post:

\begin{examples}
  \command{blaze-add}{write post in external text editor}
  \command{blaze-add release\_notes.txt}{add post from an existing file}
  \command{blaze-add first.txt second.txt}{add multiple posts at once}
  \command{blaze-add -{}-html}{force writing the post in HTML}
\end{examples}

\noindent Add new page:

\begin{examples}
  \command{blaze-add -{}-page}{write page in external text editor}
  \command{blaze-add -{}-page about.txt}{add page from an existing file}
  \command{blaze-add -{}-page about.txt contact.txt}{add multiple pages at once}
  \command{blaze-add -{}-page -{}-html}{force writing the page in HTML}
\end{examples}

\noindent Use helpful placeholders:

\begin{examples}
  \command{\%root\%}{relative path to root directory}
  \command{\%home\%}{relative path to index page}
  \command{\%page[\textmd{id}]\%}{relative path to given page}
  \command{\%post[\textmd{id}]\%}{relative path to given post}
  \command{\%tag[\textmd{name}]\%}{relative path to given tag}
\end{examples}

\section*{Editing Existing Content}

\noindent Edit a post:

\begin{examples}
  \command{blaze-edit 1}{edit the post with ID 1}
  \command{blaze-edit -{}-html 1}{force editing the post in HTML}
  \command{blaze-edit -{}-force 1}{force creating raw file if missing}
\end{examples}

\noindent Edit a page:

\begin{examples}
  \command{blaze-edit -{}-page 1}{edit the page with ID 1}
  \command{blaze-edit -{}-html -{}-page 1}{force editing the page in HTML}
  \command{blaze-edit -{}-force -{}-page 1}{force creating raw file if missing}
\end{examples}

\noindent Use helpful placeholders:

\begin{examples}
  \command{\%root\%}{relative path to root directory}
  \command{\%home\%}{relative path to index page}
  \command{\%page[\textmd{id}]\%}{relative path to given page}
  \command{\%post[\textmd{id}]\%}{relative path to given post}
  \command{\%tag[\textmd{name}]\%}{relative path to given tag}
\end{examples}

\section*{Removing Existing Content}

\noindent Remove a post:

\begin{examples}
  \command{blaze-remove 1}{remove the post with ID 1}
  \command{blaze-remove 1 2 3}{remove multiple posts at once}
  \command{blaze-remove -{}-interactive 1}{prompt before every removal}
\end{examples}

\noindent Remove a page:

\begin{examples}
  \command{blaze-remove -{}-page 1}{remove the page with ID 1}
  \command{blaze-remove -{}-page 1 2 3}{remove multiple pages at once}
  \command{blaze-remove -{}-interactive -{}-page 1}{prompt before every removal}
\end{examples}

\section*{Generating the Static Content}

\noindent Create static HTML pages:

\begin{examples}
  \command{blaze-make}{create it in the current directory}
  \command{blaze-make -{}-destdir \~{}/public\_html}{create it in the selected directory}
  \command{blaze-make -{}-verbose}{list files as they are created}
  \command{blaze-make -{}-full-paths}{make links point to index files}
  \command{blaze-make -{}-no-rss}{do not create the RSS file}
\end{examples}

\section*{Browsing the Blog Repository}

\noindent List posts:

\begin{examples}
  \command{blaze-list}{list all posts}
  \command{blaze-list -{}-short}{list each post on a single line}
  \command{blaze-list -{}-author David}{list posts written by David}
  \command{blaze-list -{}-tag fantasy}{list posts tagged as `fantasy'}
  \command{blaze-list -{}-year 2008 -{}-month 12}{list posts from December 2008}
  \command{blaze-list -{}-number 5}{list latest five posts}
  \command{blaze-list -{}-reverse}{list posts in reverse order}
\end{examples}

\noindent List pages:

\begin{examples}
  \command{blaze-list -{}-pages}{list all pages}
  \command{blaze-list -{}-pages -{}-short}{list each page on a single line}
\end{examples}

\noindent Show blog statistics:

\begin{examples}
  \command{blaze-list -{}-stats}{display all statistics}
  \command{blaze-list -{}-stats -{}-short}{display statistics on a single line}
\end{examples}

\section*{Observing the Blog History}

\noindent List repository changes:

\begin{examples}
  \command{blaze-log}{list all changes}
  \command{blaze-log -{}-short}{list each change on a single line}
  \command{blaze-log -{}-number 5}{list latest five changes}
  \command{blaze-log -{}-reverse}{list changes in reverse order}
\end{examples}

\end{multicols}

\end{document}
